% Options for packages loaded elsewhere
\PassOptionsToPackage{unicode}{hyperref}
\PassOptionsToPackage{hyphens}{url}
%
\documentclass[
]{article}
\title{php2650\_proj}
\author{Yifan Zhao}
\date{12/05/2022}

\usepackage{amsmath,amssymb}
\usepackage{lmodern}
\usepackage{iftex}
\ifPDFTeX
  \usepackage[T1]{fontenc}
  \usepackage[utf8]{inputenc}
  \usepackage{textcomp} % provide euro and other symbols
\else % if luatex or xetex
  \usepackage{unicode-math}
  \defaultfontfeatures{Scale=MatchLowercase}
  \defaultfontfeatures[\rmfamily]{Ligatures=TeX,Scale=1}
\fi
% Use upquote if available, for straight quotes in verbatim environments
\IfFileExists{upquote.sty}{\usepackage{upquote}}{}
\IfFileExists{microtype.sty}{% use microtype if available
  \usepackage[]{microtype}
  \UseMicrotypeSet[protrusion]{basicmath} % disable protrusion for tt fonts
}{}
\makeatletter
\@ifundefined{KOMAClassName}{% if non-KOMA class
  \IfFileExists{parskip.sty}{%
    \usepackage{parskip}
  }{% else
    \setlength{\parindent}{0pt}
    \setlength{\parskip}{6pt plus 2pt minus 1pt}}
}{% if KOMA class
  \KOMAoptions{parskip=half}}
\makeatother
\usepackage{xcolor}
\IfFileExists{xurl.sty}{\usepackage{xurl}}{} % add URL line breaks if available
\IfFileExists{bookmark.sty}{\usepackage{bookmark}}{\usepackage{hyperref}}
\hypersetup{
  pdftitle={php2650\_proj},
  pdfauthor={Yifan Zhao},
  hidelinks,
  pdfcreator={LaTeX via pandoc}}
\urlstyle{same} % disable monospaced font for URLs
\usepackage[margin=1in]{geometry}
\usepackage{graphicx}
\makeatletter
\def\maxwidth{\ifdim\Gin@nat@width>\linewidth\linewidth\else\Gin@nat@width\fi}
\def\maxheight{\ifdim\Gin@nat@height>\textheight\textheight\else\Gin@nat@height\fi}
\makeatother
% Scale images if necessary, so that they will not overflow the page
% margins by default, and it is still possible to overwrite the defaults
% using explicit options in \includegraphics[width, height, ...]{}
\setkeys{Gin}{width=\maxwidth,height=\maxheight,keepaspectratio}
% Set default figure placement to htbp
\makeatletter
\def\fps@figure{htbp}
\makeatother
\setlength{\emergencystretch}{3em} % prevent overfull lines
\providecommand{\tightlist}{%
  \setlength{\itemsep}{0pt}\setlength{\parskip}{0pt}}
\setcounter{secnumdepth}{-\maxdimen} % remove section numbering
\ifLuaTeX
  \usepackage{selnolig}  % disable illegal ligatures
\fi

\begin{document}
\maketitle

\textbf{Reference:}
\url{https://bmcmedresmethodol.biomedcentral.com/articles/10.1186/s12874-017-0383-8}

\hypertarget{survival-analysis}{%
\section{Survival Analysis}\label{survival-analysis}}

\hypertarget{survival-data}{%
\subsection{Survival Data}\label{survival-data}}

\begin{itemize}
\tightlist
\item
  Goal: Assess the effect of risk factors on survival
\item
  Outcome: Y\_i: time to event; U\_i Time to censoring
\end{itemize}

\hypertarget{censoring}{%
\subsection{Censoring}\label{censoring}}

Types of Censoring:

\begin{itemize}
\tightlist
\item
  Right censoring: \(T \geq T_E\)
\item
  Left censoring: \(T \leq T_0\)
\item
  Interval censoring: \(T_A \leq T \leq T_B\)
\end{itemize}

T: actual survival time\\
\(T_0\): Start\\
\(T_E\): End of followup

\hypertarget{hazard-function}{%
\subsection{Hazard Function}\label{hazard-function}}

\(H(t) = \frac {P(t\leq T<t+\Delta t|T\geq t)} {\Delta(t)}\)

Cumulative Hazard Function: \[H(t)=\int_{0}^{t} h(u) du\]

\hypertarget{cox-porportional-hazards-model}{%
\subsection{Cox porportional hazards
model}\label{cox-porportional-hazards-model}}

\begin{itemize}
\item
  Y: {[}T, C{]}

  \begin{itemize}
  \tightlist
  \item
    T: Observed survival time
  \item
    C: Censoring (or event) status
  \end{itemize}
\item
  Hazard rate:

  \(h(t)=h_0(t)exp(\sum_{j=1}^p\beta_jX_j)\) assume \(X_j\)
  time-independent.
\end{itemize}

\hypertarget{ph-assumption}{%
\subsection{PH assumption}\label{ph-assumption}}

Hazard Ratio: Assuming theta independent of time (constant over time)

\(HR=\frac{h(t,X^*)}{h(t,X)}=exp(\sum_{j=1}^p\beta_j(X_j^*-X_j))=\theta\)

\hypertarget{rsf-vs-cif-conditional-inference-forest}{%
\subsection{RSF vs CIF (conditional inference
forest)}\label{rsf-vs-cif-conditional-inference-forest}}

Survival trees and Random survival forest (RSF) models have been
identified as alternative methods to the Cox proportional hazards model
in analysing time-to-event data. (BMC Medical paper)

\begin{itemize}
\tightlist
\item
  As alternative when PH assumption violated
\item
  survival trees: fully \emph{non-parametric} - flexible, easily handle
  high dimensional covariate data

  \begin{itemize}
  \tightlist
  \item
    flexibility and can automatically detect certain types of
    \textbf{interactions} without the need to specify them beforehand
  \item
    drawback: bias towards inclusion of variables with many split points
  \end{itemize}
\end{itemize}

CIF: correct bias in RSF ``by separating the algorithm for selecting the
best covariate to split on from that of the best split point search
{[}15, 17, 18{]}.''

\emph{Cox}-proportional hazards model is usually used for right censored
time-to-event data. Mode is convenient for flexibility and simplicity,
but it's restricted to \emph{proportional hazards (PH) assumptions}.

\hypertarget{survival-random-forest-survival-tree}{%
\section{Survival Random Forest / Survival
Tree}\label{survival-random-forest-survival-tree}}

Idea: partitioning the covariate space recursively to form groups of
subjects who are similar according to the time-to-event outcome.
Minimizing a given impurity measure. Goal: To identify prognostic
factors that are predictive of the time-to-event outcome.

\hypertarget{split-rules}{%
\subsection{Split Rules:}\label{split-rules}}

\hypertarget{the-log-rank-split-rule}{%
\subsubsection{The log-rank split-rule
\ldots{}}\label{the-log-rank-split-rule}}

\ldots{} The best split at a node h, on a covariate x at a split point s
∗ is the one that gives the largest log-rank statistic between the two
daughter nodes ..

\hypertarget{the-log-rank-score-split-rule}{%
\subsubsection{The log-rank score split-rule
\ldots{}}\label{the-log-rank-score-split-rule}}

\ldots{} Trees are generally unstable and hence researchers have
recommended the growing of a collection of trees {[}10, 27{]}, commonly
referred to as random survival forests {[}20, 26{]}.

\hypertarget{cif-conditional-inference-forest}{%
\section{CIF conditional inference
forest}\label{cif-conditional-inference-forest}}

The random survival forests algorithm, has been criticised for having a
bias towards selecting variables with many split points and the
conditional inference forest algorithm has been identified as a method
to reduce this selection bias.

\hypertarget{application}{%
\section{Application}\label{application}}

\hypertarget{r}{%
\subsection{R}\label{r}}

GBSG2, pbc data from survival package

\hypertarget{python}{%
\subsection{Python}\label{python}}

\end{document}
